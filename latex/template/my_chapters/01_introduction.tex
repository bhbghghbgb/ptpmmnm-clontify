% \chapter{Introduction}
\chapter{Lời mở đầu}
\label{ch:into} % This how you label a chapter and the key (e.g., ch:into) will be used to refer this chapter ``Introduction'' later in the report. 
% the key ``ch:into'' can be used with command \ref{ch:intor} to refere this Chapter.

\begin{comment}
\textbf{Guidance on introduction chapter writing:} Introductions are written in the following parts:
\begin{itemize}
    \item A brief  description of the investigated problem.
    \item A summary of the scope and context of the project, i.e., what is the background of the topic/problem/application/system/algorithm/experiment/research question/hypothesis/etc. under investigation/implementation/development [whichever is applicable to your project].
    \item The aims and objectives of the project.
    \item A description of the problem and the methodological approach adopted to solve the problem.
    \item A summary of the most significant outcomes and their interpretations.
    \item Organization of the report. 
\end{itemize}


Consult \textbf{your supervisor} to check the content of the introduction chapter. In this template, we only offer basic sections of an introduction chapter. It may  not be complete and comprehensive. Writing a report is a subjective matter, and a report's style and structure depend on the ``type of project'' as well as an individual's preference. This template suits the following project paradigms:
\begin{enumerate}
    \item software engineering and software/web application development;
    \item algorithm implementation, analysis and/or application;  
    \item science lab (experiment); and
    \item pure theoretical development (not mention extensively).
\end{enumerate}

Use only a single \textbf{font} for the body text. We recommend using a clean and electronic document friendly font like \textbf{Arial} or \textbf{Calibri} for MS-word (If you create a report in MS word). If you use this template, DO NOT ALTER the template's default font ``amsfont default computer modern''. The default \LaTeX~font ``computer modern'' is also acceptable. 

The recommended body text \textbf{font size} is minimum \textbf{11pt} and minimum one-half line spacing. The recommended figure/table caption font size is minimum 10pt. The footnote\footnote{Example footnote: footnotes are useful for adding external sources such as links as well as extra information on a topic or word or sentence. Use command \textbackslash footnote\{...\} next to a word to generate a footnote in \LaTeX.} font size is minimum 8pt. DO NOT ALTER the font setting of this template. 
\end{comment}

%%%%%%%%%%%%%%%%%%%%%%%%%%%%%%%%%%%%%%%%%%%%%%%%%%%%%%%%%%%%%%%%%%%%%%%%%%%%%%%%%%%
\begin{comment}
\section{Background}
\label{sec:into_back}
Describe to a reader the context of your project. That is, what is your project and what its motivation. Briefly explain the major theories, applications, and/or products/systems/algorithms whichever is relevant to your project.

\textbf{Cautions:} Do not say you choose this project because of your interest, or your supervisor proposed/suggested this project, or you were assigned this project as your final year project. This all may be true, but it is not meant to be written here.

%%%%%%%%%%%%%%%%%%%%%%%%%%%%%%%%%%%%%%%%%%%%%%%%%%%%%%%%%%%%%%%%%%%%%%%%%%%%%%%%%%%
\section{Problem statement}
\label{sec:intro_prob_art}
This section describes the investigated problem in detail. You can also have a separate chapter on ``Problem articulation.''  For some projects, you may have a section like ``Research question(s)'' or ``Research Hypothesis'' instead of a section on ``Problem statement.'

%%%%%%%%%%%%%%%%%%%%%%%%%%%%%%%%%%%%%%%%%%%%%%%%%%%%%%%%%%%%%%%%%%%%%%%%%%%%%%%%%%%
\section{Aims and objectives}
\label{sec:intro_aims_obj}
Describe the ``aims and objectives'' of your project. 

\textbf{Aims:} The aims tell a reader what you want/hope to achieve at the end of the project. The  aims define your intent/purpose in general terms.  

\textbf{Objectives:} The objectives are a set of tasks you would perform in order to achieve the defined aims. The objective statements have to be specific and measurable through the results and outcome of the project.



%%%%%%%%%%%%%%%%%%%%%%%%%%%%%%%%%%%%%%%%%%%%%%%%%%%%%%%%%%%%%%%%%%%%%%%%%%%%%%%%%%%
\section{Solution approach}
\label{sec:intro_sol} % label of Org section
Briefly describe the solution approach and the methodology applied in solving the set aims and objectives.

Depending on the project, you may like to alter the ``heading'' of this section. Check with you supervisor. Also, check what subsection or any other section that can be added in or removed from this template.

\subsection{A subsection 1}
\label{sec:intro_some_sub1}
You may or may not need subsections here. Depending on your project's needs, add two or more subsection(s). A section takes at least two subsections. 

\subsection{A subsection 2}
\label{sec:intro_some_sub2}
Depending on your project's needs, add more section(s) and subsection(s).

\subsubsection{A subsection 1 of a subsection}
\label{sec:intro_some_subsub1}
The command \textbackslash subsubsection\{\} creates a paragraph heading in \LaTeX.

\subsubsection{A subsection 2 of a subsection}
\label{sec:intro_some_subsub2}
Write your text here...

%%%%%%%%%%%%%%%%%%%%%%%%%%%%%%%%%%%%%%%%%%%%%%%%%%%%%%%%%%%%%%%%%%%%%%%%%%%%%%%%%%%
\section{Summary of contributions and achievements} %  use this section 
\label{sec:intro_sum_results} % label of summary of results
Describe clearly what you have done/created/achieved and what the major results and their implications are. 


%%%%%%%%%%%%%%%%%%%%%%%%%%%%%%%%%%%%%%%%%%%%%%%%%%%%%%%%%%%%%%%%%%%%%%%%%%%%%%%%%%%
\section{Organization of the report} %  use this section
\label{sec:intro_org} % label of Org section
Describe the outline of the rest of the report here. Let the reader know what to expect ahead in the report. Describe how you have organized your report. 

\textbf{Example: how to refer a chapter, section, subsection}. This report is organised into seven chapters. Chapter~\ref{ch:lit_rev} details the literature review of this project. In Section~\ref{ch:method}...  % and so on.
\end{comment}

\section{Bối cảnh}
\label{sec:intro_back}
Trong bối cảnh sự phát triển mạnh mẽ của âm nhạc trực tuyến và nhu cầu giải trí đa dạng của người dùng, các nền tảng phát nhạc trực tuyến như Spotify đã trở thành một phần không thể thiếu trong cuộc sống số. Với khả năng truy cập hàng triệu bài hát, podcast và video âm nhạc mọi lúc mọi nơi, những ứng dụng này mang lại trải nghiệm âm nhạc liền mạch và cá nhân hóa.

Dự án này tập trung vào việc phát triển một phần mềm mô phỏng Spotify (Spotify Clone) nhằm khám phá các công nghệ web hiện đại và xây dựng một ứng dụng phát nhạc trực tuyến cơ bản. Dự án sẽ đi sâu vào việc thiết kế giao diện người dùng trực quan, xây dựng các chức năng phát nhạc và video, quản lý thư viện cá nhân của người dùng, cũng như cung cấp một trang quản trị hệ thống. Các công nghệ chủ yếu được sử dụng trong dự án bao gồm ngôn ngữ lập trình Python, framework backend Django, thư viện frontend React, hệ quản trị cơ sở dữ liệu PostgreSQL, hạ tầng đám mây AWS để host ứng dụng, dịch vụ lưu trữ S3, và giao thức gRPC với Protobuf cho giao tiếp microservice.

\section{Phát biểu bài toán}
\label{sec:intro_prob_art}
Bài toán đặt ra là xây dựng một ứng dụng web Spotify Clone có khả năng cung cấp các chức năng phát nhạc và video âm nhạc cơ bản, cho phép người dùng tạo và quản lý thư viện nhạc cá nhân (album, bài hát yêu thích), đồng thời cung cấp một giao diện quản trị cho phép quản lý hệ thống. Ứng dụng cần đảm bảo khả năng hoạt động ổn định và có khả năng mở rộng thông qua kiến trúc microservice và triển khai trên môi trường đám mây.

Các khía cạnh cụ thể cần được giải quyết bao gồm:
\begin{itemize}
    \item Thiết kế giao diện người dùng (frontend) thân thiện và dễ sử dụng cho các chức năng phát nhạc, video, quản lý thư viện và chat bằng React.
    \item Xây dựng hệ thống backend bằng Python và Django theo kiến trúc MVC kết hợp Controller/Service/Repository và microservice để xử lý các yêu cầu từ frontend, quản lý dữ liệu âm nhạc, video và người dùng.
    \item Thiết kế và xây dựng cơ sở dữ liệu PostgreSQL để lưu trữ thông tin về bài hát, video, album, người dùng và các dữ liệu liên quan khác.
    \item Hiện thực hóa các chức năng phát nhạc, phát video, tải video (tùy chọn), tạo và quản lý thư viện cá nhân, và trang quản trị.
    \item Triển khai ứng dụng trên hạ tầng AWS và sử dụng S3 cho lưu trữ nội dung.
    \item Sử dụng gRPC và Protobuf cho giao tiếp hiệu quả giữa các microservice.
    \item Tích hợp chức năng chat trực tiếp vào giao diện web.
\end{itemize}

\section{Mục tiêu và nhiệm vụ}
\label{sec:intro_aims_obj}
\textbf{Mục tiêu:} Mục tiêu chính của dự án là xây dựng thành công một phần mềm Spotify Clone cơ bản, có khả năng trình diễn các chức năng phát nhạc, video âm nhạc, quản lý thư viện người dùng và cung cấp giao diện quản trị trên nền tảng web, sử dụng các công nghệ Python, Django (theo kiến trúc MVC và microservice), React, PostgreSQL và triển khai trên AWS.

\textbf{Nhiệm vụ:} Để đạt được mục tiêu trên, nhóm chúng em sẽ thực hiện các nhiệm vụ cụ thể sau:
\begin{itemize}
    \item Nghiên cứu và phân tích yêu cầu của đề bài, tìm hiểu về kiến trúc và các tính năng của Spotify.
    \item Lựa chọn và làm quen với các công nghệ phát triển (Python, Django, React, PostgreSQL, AWS, Docker, gRPC, Protobuf).
    \item Thiết kế cơ sở dữ liệu PostgreSQL.
    \item Thiết kế kiến trúc tổng thể của ứng dụng theo mô hình MVC và microservice, bao gồm việc container hóa các service bằng Docker.
    \item Thiết kế giao diện người dùng (wireframe và mockup) bằng React.
    \item Phát triển các microservice backend (API) bằng Python và Django theo kiến trúc Controller/Service/Repository cho việc phát nhạc, video, quản lý người dùng và thư viện.
    \item Phát triển giao diện người dùng (frontend) bằng React tương tác với các microservice backend.
    \item Hiện thực hóa trang quản trị hệ thống.
    \item Triển khai và quản lý ứng dụng trên hạ tầng AWS, sử dụng S3 cho lưu trữ nội dung.
    \item Thiết lập giao tiếp giữa các microservice bằng gRPC và Protobuf.
    \item Nghiên cứu và tích hợp chức năng chat vào ứng dụng.
    \item Kiểm thử và gỡ lỗi phần mềm.
    \item Viết báo cáo chi tiết về quá trình phát triển và kết quả của dự án.
    \item Chuẩn bị tài liệu hướng dẫn cài đặt và sử dụng (README.md).
\end{itemize}

\section{Phương pháp tiếp cận}
\label{sec:intro_sol}
Dự án này sẽ được triển khai theo phương pháp phát triển phần mềm Waterfall cho các giai đoạn chính, kết hợp với mô hình MVC (Model-View-Controller) trong kiến trúc Django backend và kiến trúc microservice với việc sử dụng Docker để container hóa các dịch vụ. Chúng em sẽ tập trung vào việc phân tách ứng dụng thành các dịch vụ nhỏ độc lập, giao tiếp với nhau thông qua giao thức gRPC và Protobuf, nhằm đảm bảo tính linh hoạt, khả năng mở rộng và dễ bảo trì của hệ thống.

\subsection{Công nghệ sử dụng}
\label{sec:intro_tech}
Các công nghệ chính được sử dụng trong dự án bao gồm:
\begin{itemize}
    \item \textbf{Ngôn ngữ lập trình Backend:} Python 3
    \item \textbf{Framework Backend:} Django (theo kiến trúc MVC và microservice)
    \item \textbf{Ngôn ngữ lập trình Frontend:} JavaScript
    \item \textbf{Framework Frontend:} React
    \item \textbf{Hệ quản trị cơ sở dữ liệu:} PostgreSQL
    \item \textbf{Hạ tầng đám mây:} Amazon Web Services (AWS)
    \item \textbf{Lưu trữ nội dung:} Amazon S3
    \item \textbf{Container hóa:} Docker
    \item \textbf{Giao tiếp Microservice:} gRPC
    \item \textbf{Serialization:} Protobuf
    \item \textbf{Công cụ quản lý phiên bản:} Git
    \item \textbf{Nền tảng cộng tác phát triển:} GitHub
\end{itemize}

\subsection{Quy trình phát triển}
\label{sec:intro_process}
Quy trình phát triển dự kiến sẽ bao gồm các giai đoạn chính theo mô hình Waterfall:
\begin{enumerate}
    \item Phân tích yêu cầu và lập kế hoạch chi tiết.
    \item Thiết kế cơ sở dữ liệu PostgreSQL và kiến trúc microservice.
    \item Thiết kế giao diện người dùng (UI/UX) bằng React.
    \item Phát triển các microservice backend (API) bằng Django theo kiến trúc Controller/Service/Repository.
    \item Phát triển frontend bằng React và tích hợp với các microservice.
    \item Triển khai và container hóa các microservice bằng Docker, triển khai trên AWS và cấu hình S3.
    \item Thiết lập giao tiếp gRPC giữa các microservice.
    \item Kiểm thử và gỡ lỗi toàn bộ hệ thống.
    \item Viết tài liệu và báo cáo chi tiết.
    \item Đóng gói và chuẩn bị tài liệu hướng dẫn cài đặt và sử dụng (README.md).
\end{enumerate}

\begin{comment}
\section{Tóm tắt các đóng góp và thành tựu}
\label{sec:intro_sum_results}
Khi hoàn thành dự án này, chúng em kỳ vọng sẽ xây dựng được một ứng dụng web Spotify Clone có khả năng:
\begin{itemize}
    \item Phát và quản lý danh sách nhạc, video âm nhạc.
    \item Cho phép người dùng tạo và quản lý album, bài hát yêu thích.
    \item Cung cấp một trang quản trị đơn giản để quản lý nội dung và người dùng.
    \item Được xây dựng theo kiến trúc microservice, dễ dàng mở rộng và bảo trì.
    \item Được triển khai trên hạ tầng đám mây AWS, đảm bảo tính ổn định và khả năng truy cập.
    \item Sử dụng các công nghệ hiện đại như Django, React, PostgreSQL, Docker và gRPC.
    \item (Tùy chọn) Tích hợp chức năng chat trực tiếp giữa người dùng.
\end{itemize}
Thành tựu chính của dự án sẽ là việc ứng dụng thành công các kiến thức về phát triển web (frontend và backend), quản lý cơ sở dữ liệu, kiến trúc microservice, container hóa và triển khai đám mây để tạo ra một sản phẩm có tính thực tế, có khả năng mở rộng và trình diễn được.

\section{Cấu trúc báo cáo}
\label{sec:intro_org}
Báo cáo này được tổ chức thành năm chương. Chương \ref{sec:lit_rev} sẽ trình bày cơ sở lý thuyết và các công nghệ liên quan đến dự án, bao gồm Django, React, PostgreSQL, kiến trúc microservice, Docker và AWS. Chương \ref{sec:design} sẽ mô tả chi tiết về thiết kế cơ sở dữ liệu PostgreSQL, kiến trúc tổng thể của ứng dụng theo mô hình MVC và microservice, cũng như thiết kế giao diện người dùng React. Chương \ref{sec:implementation} sẽ tập trung vào quá trình hiện thực hóa các microservice backend bằng Django và frontend bằng React, cũng như việc tích hợp chúng. Kết quả kiểm thử và đánh giá hiệu năng của hệ thống sẽ được trình bày trong Chương \ref{sec:evaluation}. Cuối cùng, Chương \ref{sec:conclusion} sẽ đưa ra kết luận về dự án và các hướng phát triển tiềm năng.
\end{comment}

\begin{comment}
\textbf{Lưu ý:} Hãy chú ý sử dụng đúng các từ như "Chương", "Mục", "Hình" trước lệnh \textbackslash ref\{\} để đảm bảo tính rõ ràng của văn bản. Ví dụ, "Trong Chương \ref{sec:lit_rev}..." thay vì "Trong \ref{sec:lit_rev}...".
\end{comment}

\begin{comment}
\textbf{Note:}  Take care of the word like ``Chapter,'' ``Section,'' ``Figure'' etc. before the \LaTeX~command \textbackslash ref\{\}. Otherwise, a  sentence will be confusing. For example, In \ref{ch:lit_rev} literature review is described. In this sentence, the word ``Chapter'' is missing. Therefore, a reader would not know whether 2 is for a Chapter or a Section or a Figure.  For more information on \textbf{automated tools} to assist in this work, see \Cref{subsec:reftools}.
\end{comment}

